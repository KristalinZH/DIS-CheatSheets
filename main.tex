\documentclass[a4paper,12pt]{extarticle}

\usepackage[utf8]{inputenc}
\usepackage[bulgarian]{babel} 
\usepackage{amsmath} 
\usepackage{amssymb} 
\usepackage{geometry} 
\geometry{top=0.5cm, left=2cm, right=2cm}
\usepackage{multicol} 
\usepackage{xcolor}

\DeclareMathOperator*\uplim{\overline{lim}}

\newcommand{\entry}[2]{\textbf{#1 = } #2 \\}
\newcommand{\lgt}{\fontsize{18pt}{20pt}\selectfont}
\newcommand{\mdt}{\fontsize{15pt}{15pt}\selectfont}
\newcommand{\smt}{\fontsize{13pt}{13pt}\selectfont}

\begin{document}

\section*{\lgt\centering{Основни граници}}
\begin{multicols}{3}
\smt{\[\lim_{x \to 0} \frac{\sin x}{x} = 1\]}
\smt{\[\lim_{t(x) \to 0} \frac{\sin t(x)}{t(x)} = 1\]}
\smt{\[\lim_{x \to 0} \frac{\arcsin x}{x} = 1\]}
\smt{\[\lim_{x \to 0} \frac{\tg x}{x} = 1\]}
\smt{\[\lim_{t(x) \to 0} \frac{\tg t(x)}{t(x)} = 1\]}
\smt{\[\lim_{x \to 0} \frac{\arctg x}{x} = 1\]}
\smt{\[\lim_{x \to 0} \frac{\ln (1+x)}{x} = 1\]}
\smt{\[\lim_{x \to 0} \frac{(1+x)^p - 1}{x} = p\]}
\smt{\[\lim_{x \to 0} \frac{a^x-1}{x} = \ln a\]}
\smt{\[\lim_{x \to 0} \frac{e^x-1}{x} = 1\]}
\smt{\[\lim_{x \to \infty} (1+\frac{1}{x})^x = e\]}
\smt{\[\lim_{x \to 0} (1+x)^\frac{1}{x} = e\]}
\end{multicols}

\begin{center}
\vspace{-1em}
\smt{\[\ln(1+x) \sim x+ o(x)\]}
\vspace{-1em}
\smt{\[\sin x \sim x+ o(x)\]}
\end{center}

\section*{\lgt\centering{Критерии за сходимост на редове}}
\vspace{1em}
\begin{multicols}{3}

\subsection*{\mdt\centering{Критерий на Даламбер}}
\vspace{-1em}
\begin{center}
\mdt{\[\lim_{n \to \infty} \frac{a_{n+1}}{a_n} = l\]}
\smt{\[\textbf{l}:
\left\{
\begin{array}{l}
\smt{l<1, \hspace{0.2em} \textbf{сходящ}} \\
\smt{l>1, \hspace{0.2em} \textbf{разходящ}} \\
\smt{l=1, \hspace{0.2em} \textbf{неопределеност}}
\end{array}
\right.
\]}
\end{center}

\subsection*{\mdt\centering{Критерий на Коши}}
\vspace{-1em}
\begin{center}
\mdt{\[\lim_{n \to \infty} \sqrt[n]{a_n} = l\]}
\smt{\[\textbf{l}:
\left\{
\begin{array}{l}
\smt{l<1, \hspace{0.2em} \textbf{сходящ}} \\
\smt{l>1, \hspace{0.2em} \textbf{разходящ}} \\
\smt{l=1, \hspace{0.2em} \textbf{неопределеност}}
\end{array}
\right.
\]}
\end{center}

\subsection*{\mdt\centering{Критерий на Раабе-Дюамел}}
\vspace{-1em}
\begin{center}
\mdt{\[\lim_{n \to \infty} n(\frac{a_n}{a_{n+1}}-1) = l\]}
\smt{\[\textbf{l}:
\left\{
\begin{array}{l}
\smt{l>1, \hspace{0.2em} \textbf{сходящ}} \\
\smt{l<1, \hspace{0.2em} \textbf{разходящ}} \\
\smt{l=1, \hspace{0.2em} \textbf{неопределеност}}
\end{array}
\right.
\]}
\end{center}

\end{multicols}

\section*{\lgt\centering{Основни производни}}
\begin{multicols}{3}
\smt{\[(const)' = 0\]}
\smt{\[(x^n)' = nx^{n-1}\]}
\smt{\[(a^x)' = a^x\ln a\]}
\smt{\[(\log_a x)' = \frac{1}{x\ln a}\]}
\smt{\[(e^x)' = e^x\]}
\smt{\[(\ln x)' = \frac{1}{x}\]}
\smt{\[(\sin x)' = \cos x\]}
\smt{\[(\cos x)' = -\sin x\]}
\smt{\[(\tg x)' = \frac{1}{\cos^2 x}\]}
\smt{\[(\ctg x)' = -\frac{1}{\sin^2 x}\]}
\smt{\[(\arcsin x)' = \frac{1}{\sqrt{1-x^2}}\]}
\smt{\[(\arccos x)' = -\frac{1}{\sqrt{1-x^2}}\]}
\smt{\[(\arctg x)' = \frac{1}{1+x^2}\]}
\smt{\[(\arcctg x)' = -\frac{1}{1+x^2}\]}
\end{multicols}

\section*{\lgt\centering{Прозиводни от n-ти ред}}
\vspace{-1em}
\begin{center}
\smt{\[(\sin x)^{[n]} = \sin (x+n\frac{\pi}{2})\]}
\smt{\[(a^x)^{[n]} = a^x\ln^n a\]}
\smt{\[(\ln x)^{[n]} = (-1)^{n-1}(n-1)!\frac{1}{x^n}\]}
\smt{\[(x^a)^{[n]}:\quad\quad 0<a<n \Rightarrow  0\quad\quad a=n \Rightarrow  n!\quad\quad a>n \Rightarrow  a(a-1)(a-2)\dots(a-n+1)x^{a-n}\]}
\smt\centering{\[a<0 \Rightarrow (-1)^na(a+1)(a+2)\dots(a+n-1)\frac{1}{x^{a+n}}\]}
\end{center}

\section*{\lgt\centering{Формула на Лайбниц}}
\vspace{-1em}
\begin{center}
\smt\centering{\[(f(x)*g(x))^{[n]} = \sum_{k=0}^{n} \binom{n}{k}f(x)^{[n-k]}g(x)^{[k]}\]}
\end{center}

\section*{\lgt\centering{Таблчини неопределени интеграли}}
\begin{multicols}{3}
\smt{\[\int e^x \, dx = e^x + c\]}
\smt{\[\int x^n \, dx = \frac{x^{n+1}}{n+1} + c \quad n\neq-1\]}
\smt{\[\int \sin x \, dx = -\cos x + c\]}
\smt{\[\int \cos x \, dx = \sin x + c\]}
\smt{\[\int \frac{1}{\cos^2 x} \, dx = \tg x + c\]}
\smt{\[\int -\frac{1}{\sin^2 x} \, dx = \ctg x + c\]}
\smt{\[\int \frac{1}{\sqrt{1-x^2}} \, dx = \arcsin x + c\]}
\smt{\[\int -\frac{1}{\sqrt{1-x^2}} \, dx = \arccos x + c\]}
\smt{\[\int \frac{1}{1+x^2} \, dx = \arctg x + c\]}
\smt{\[\int -\frac{1}{1+x^2} \, dx = \arcctg x + c\]}
\smt{\[\int \frac{1}{x} \, dx = \ln\left|x\right| + c\]}
\smt{\[\int \frac{1}{\sqrt{x^2 \pm a^2}} \, dx = \ln\left|x+\sqrt{x^2 \pm a^2}\right| + c\]}
\end{multicols}

\section*{\lgt\centering{Интегриране по части}}
\vspace{-1em}
\begin{center}
\smt\centering{\[\int f(x) \, d(g(x)) = f(x)*g(x) -\int g(x) \, d(f(x))\]}
\end{center}

\section*{\lgt\centering{Основни свойства на определени интеграли}}
\vspace{-1em}
\begin{center}
\smt{\[1. \hspace{0.2em} \int_a^b (\alpha f(x) \pm \beta g(x)) \, dx \hspace{0.2em} = \hspace{0.2em} \alpha\int_a^b f(x) \, dx \hspace{0.2em} \pm \hspace{0.2em} \beta\int_a^b g(x) \, dx\]}
\smt{\[2. \hspace{0.2em} \int_a^b f(x) \, dx \hspace{0.2em} = \hspace{0.2em} \int_a^c f(x) \, dx \hspace{0.2em} + \hspace{0.2em} \int_c^b f(x) \, dx \qquad c \in [a,b]\]}
\smt{\[3. \hspace{0.2em} f(x) \leq g(x) \hspace{0.2em} \forall x \in [a,b] \hspace{0.2em} \Rightarrow \hspace{0.2em} \int_a^b f(x) \, dx \hspace{0.2em} \leq \hspace{0.2em} \int_a^b g(x) \, dx\]}
\smt{\[4. \hspace{0.2em} \int_a^a f(x) \, dx \hspace{0.2em} = \hspace{0.2em} 0\]}
\smt{\[5. \hspace{0.2em} \int_a^b f(x) \, dx \hspace{0.2em} = \hspace{0.2em} -\int_b^a f(x) \, dx\]}
\smt{\[6. \hspace{0.2em} \left|\int_a^b f(x) \, dx\right| \hspace{0.2em} \leq \hspace{0.2em} \int_a^b \left|f(x)\right| \, dx\]}
\end{center}

\section*{\lgt\centering{Интегриране по части на определен интеграл}}
\vspace{-1em}
\begin{center}
\smt\centering{\[\int_a^b f(x) \, d(g(x)) \hspace{0.2em} = \hspace{0.2em} f(x)*g(x)\Big|_a^b \hspace{0.2em} - \hspace{0.2em} \int_a^b g(x) \, d(f(x))\]}
\end{center}

\section*{\lgt\centering{Основни формули на определени интеграли}}
\vspace{-1em}
\begin{center}
\smt\centering{\[1. \hspace{0.2em} f(x) \text{ е} \hspace{0.2em}  \textbf{непрекъсната} \hspace{0.2em}  \text{в }[-a,a] \hspace{0.2em} \text{и} \hspace{0.2em}  \textbf{четна} \text{. Тогава: } \hspace{0.2em} \int_{-a}^a f(x) \, dx \hspace{0.2em} = \hspace{0.2em} 2\int_0^a f(x) \, dx\]}
\smt\centering{\[2. \hspace{0.2em} f(x) \text{ е} \hspace{0.2em}  \textbf{непрекъсната} \hspace{0.2em}  \text{в }[-a,a] \hspace{0.2em} \text{и} \hspace{0.2em}  \textbf{нечетна} \text{. Тогава: } \hspace{0.2em} \int_{-a}^a f(x) \, dx \hspace{0.2em} = \hspace{0.2em} 0\]}
\smt\centering{\[3. \hspace{0.2em} f(x) \text{ е} \hspace{0.2em} \textbf{периодична с период T} \text{. Тогава: } \hspace{0.2em} \int_a^{a+T} f(x) \, dx \hspace{0.2em} = \hspace{0.2em} \int_0^T f(x) \, dx\]}
\smt\centering{\[4. \hspace{0.2em} f(x) \text{ е} \hspace{0.2em}  \textbf{непрекъсната} \hspace{0.2em}  \text{в }[0,1] \text{. Тогава: } \hspace{0.2em} \int_{0}^{\frac{\displaystyle \pi}{2}} f(\sin x) \, dx \hspace{0.2em} = \hspace{0.2em} \int_{0}^{\frac{\displaystyle \pi}{2}} f(\cos x) \, dx\]}
\smt\centering{\[5. \hspace{0.2em} f(x) \text{ е} \hspace{0.2em}  \textbf{непрекъсната} \hspace{0.2em}  \text{в }[0,1] \text{. Тогава: } \hspace{0.2em} \int_{0}^{\displaystyle \pi} xf(\sin x) \, dx \hspace{0.2em} = \hspace{0.2em} \frac{\pi}{2} \int_{0}^{\displaystyle \pi} f(\sin x) \, dx\]}
\end{center}

\section*{\lgt\centering{Геомтетрични приложения на определения интеграл}}
\vspace{-2em}
\begin{center}
\lgt\centering{\[\textbf{Лице}\]}
\smt\centering{\[f(x) \hspace{0.2em} \text{и} \hspace{0.2em} g(x) \hspace{0.2em} \text{са } \textbf{непрекъснати} \text{ в} \hspace{0.2em} [a,b] \hspace{0.2em} \text{и} \hspace{0.2em} f(x) \leq g(x) \hspace{0.2em} \text{за всяко} \hspace{0.2em} x \hspace{0.2em} \text{в} \hspace{0.2em} [a,b] \text{. Тогава:}\]}
\smt\centering{\[S \hspace{0.2em} = \hspace{0.2em} \int_{a}^{b} (g(x) - f(x)) \, dx\]}
\mdt\centering{\[\textbf{Полярни координати}\]}
\smt\centering{\[\rho = f(\theta) \hspace{0.2em} \text{е} \hspace{0.2em} \textbf{непрекъсната} \text{ и } \textbf{неотрицателна} \text{ за } \theta \hspace{0.2em} \text{в} \hspace{0.2em} [\alpha,\beta] \text{. Тогава:}\]}
\smt\centering{\[S = \frac{1}{2} \int_{\alpha}^{\beta} f(\theta)^2 \, d\theta\]}
\vspace{1em}
\lgt\centering{\[\textbf{Дължина на крива}\]}
\smt\centering{\[x = f(t) \hspace{0.2em} \text{и} \hspace{0.2em} y  =  g(t) \hspace{0.2em} \text{са } \textbf{непрекъснати} \text{ и } \textbf{диференцируеми} \text{ за} \hspace{0.2em} t \hspace{0.2em} \text{в} \hspace{0.2em} [a,b] \text{. Тогава:} \]}
\smt\centering{\[l \hspace{0.2em} = \hspace{0.2em} \int_{a}^{b} \sqrt{f'(t)^2+g'(t)^2} \, dt\]}
\mdt\centering{\[\textbf{Графика на функция}\]}
\smt\centering{\[x = t \hspace{0.2em} \text{и} \hspace{0.2em} y  =  f(x) \hspace{0.2em} \text{са } \textbf{непрекъснати} \text{ и } \textbf{диференцируеми} \text{ за} \hspace{0.2em} t \hspace{0.2em} \text{в} \hspace{0.2em} [a,b] \text{. Тогава:} \]}
\smt\centering{\[l \hspace{0.2em} = \hspace{0.2em} \int_{a}^{b} \sqrt{1+f'(t)^2} \, dt\]}
\lgt\centering{\[\textbf{Обем}\]}
\smt\centering{\[f(x) \hspace{0.2em} \text{e } \textbf{непрекъсната} \text{ в} \hspace{0.2em} [a,b] \text{. Тогава:} \]}
\smt\centering{\[V \hspace{0.2em} = \hspace{0.2em} \pi \int_{a}^{b} f(x)^2 \, dt\]}
\end{center}

\section*{\lgt\centering{Основни несобствени интеграли за сравняване}}
\begin{multicols}{3}

\subsection*{\mdt\centering{Първи род}}
\vspace{-1em}
\begin{center}
\mdt{\[\int_{a}^{+\infty}\frac{1}{x^\lambda} \, dx \hspace{0.8em} (a>0)\]}
\mdt{\[\int_{-\infty}^{b}\frac{1}{x^\lambda} \, dx \hspace{0.8em} (b<0)\]}
\smt{\[\text{са}:
\left\{
\begin{array}{l}
\smt{\textbf{сходящи} \hspace{0.4em}\lambda>1} \\
\smt{\textbf{разходящи} \hspace{0.4em}\lambda\leq1} 
\end{array}
\right.
\]}
\end{center}

\subsection*{\mdt\centering{Втори род}}
\vspace{-1em}
\begin{center}
\mdt{\[\int_{a}^{b}\frac{1}{(x-a)^\lambda} \, dx\]}
\mdt{\[\int_{a}^{b}\frac{1}{(b-x)^\lambda} \, dx\]}
\smt{\[\text{са}:
\left\{
\begin{array}{l}
\smt{\textbf{сходящи} \hspace{0.4em}\lambda<1} \\
\smt{\textbf{разходящи} \hspace{0.4em}\lambda\geq1} 
\end{array}
\right.
\]}
\end{center}

\subsection*{\mdt\centering{Първи род}}
\vspace{-1em}
\begin{center}
\mdt{\[\int_{1}^{+\infty}\frac{1}{x^p\ln^qx} \, dx\]}
\smt{\[\text{е}:
\left\{
\begin{array}{l}
\smt{\textbf{сходящ} \hspace{0.4em}p>1} \\
\smt{\textbf{сходящ} \hspace{0.4em}p=1 \hspace{0.2em} \text{и} \hspace{0.2em} q>1} \\
\smt{\textbf{разходящ} \hspace{0.4em}p=1 \hspace{0.2em} \text{и} \hspace{0.2em} q\leq1} \\
\smt{\textbf{разходящ} \hspace{0.4em}p<1} 
\end{array}
\right.
\]}
\end{center}

\end{multicols}

\section*{\lgt\centering{Формули за радиус на сходимост на степенни редове}}
\begin{multicols}{2}

\subsection*{\mdt\centering{Формула на Даламбер}}
\vspace{-1em}
\begin{center}
\mdt{\[R = \lim_{n \to \infty} \left| \frac{a_n}{a_{n+1}} \right|\]}
\end{center}

\subsection*{\mdt\centering{Формула на Коши-Адамар}}
\vspace{-1em}
\begin{center}
\mdt{\[R = \frac{1}{\displaystyle \overline{\lim_{n \to \infty}} \sqrt[n]{\left| a_n \right|}}\]}
\end{center}

\end{multicols}

\end{document}